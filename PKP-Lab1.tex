% !TEX program = xelatex

\documentclass{VUMIFPSkursinis}
\usepackage{algorithmicx}
\usepackage{algorithm}
\usepackage{algpseudocode}
\usepackage{amsfonts}
\usepackage{amsmath}
\usepackage{bm}
\usepackage{caption}
\usepackage{color}
\usepackage{float}
\usepackage{graphicx}
\usepackage{listings}
\usepackage{float}
\usepackage{subfig}
\usepackage{wrapfig}
\usepackage[hidelinks]{hyperref}
\usepackage{todonotes}
\usepackage{lineno}

% Titulinio aprašas
\university{Vilniaus universitetas}
\faculty{Matematikos ir informatikos fakultetas}
\department{}
\papertype{Programų sistemų inžinerijos modeliai ir metodai laboratorinis darbas 2}
\title{Requirements modeling}
\titleineng{Reikalavimų modeliavimas}
\status{1 course students}
\author{Matas Savickis}
\secondauthor{Vytautas Krivickas}
\thirdauthor{Šarūnas Kazimieras Buteikis}


\supervisor{Audronė Lupeikienė, M. Darbuot., Dr}
\date{Vilnius – \the\year}

% Nustatymai
% \setmainfont{Palemonas}   % Pakeisti teksto šriftą į Palemonas (turi būti įdiegtas sistemoje)
\bibliography{bibliografija}

\begin{document}
\selectlanguage{english}
\maketitle

\tableofcontents

\section{NFR type catalogue}

\begin{figure}[htbp]
	\includegraphics[scale=0.6]{img/1}
	\caption{NFR diagram} % Antraštė įterpiama po paveikslėlio
	\label{img:kurimoProcesas}
\end{figure}

	\begin{itemize}
		\item{Virus registration time - to ensure quick response time system must provide quick to register new cases. }
		\item{Patient notification time - it is important to prevent spread of virus to imidiatley notify and isolate contagios people.}
		\item{Holding medical records - to ensure that correct treatment is applied we need to have all medical records. }
		\item{Holding foreign countries data - to check which countries are infected we need to keep up to date information about them to make decisions.}
		\item{Reliability - tracking virus must me ensured 24/7 to not miss any crutial data.}
		\item{Distributivity - system must be working in many regions in Lithuania to ensure that if one region fails other can still provide service.}
		\item{Confidentiality - system must treat sensitive person information with respect to ensure systems credability.}
		\item{Completness - all data must not be corupted and must be complete to not lose information about virus.}
		\item{Minimality - only required amount of data must be stored to ensure maximum security in case of data breach.}
		\item{Summarizability - all data must be sumarized correctly.} 
		\item{Domain Compliance - domain must be modeled compiently.}
		\item{Tracebility - to ensure }
		\item{LTU law}
		\item{GDPR}
		\item{Replication}
		\item{Tolerance algorithm}
	\end{itemize}

\section{Modelling of the non-functional requirements}

\section{Identifying and modelling of possible operationalizations for NFR}

\section{Detecting and modelling of implicit interdependencies among NFR}

\section{Chosen operationalizations}

\section{Strategic rationale model}

\section{Conclusions about an actor dependency}

\sectionnonum{Conclusions}

\end{document}
